% Options for packages loaded elsewhere
\PassOptionsToPackage{unicode}{hyperref}
\PassOptionsToPackage{hyphens}{url}
%
\documentclass[
]{article}
\title{15-Otras distribuciones}
\author{Adrian}
\date{25/1/2022}

\usepackage{amsmath,amssymb}
\usepackage{lmodern}
\usepackage{iftex}
\ifPDFTeX
  \usepackage[T1]{fontenc}
  \usepackage[utf8]{inputenc}
  \usepackage{textcomp} % provide euro and other symbols
\else % if luatex or xetex
  \usepackage{unicode-math}
  \defaultfontfeatures{Scale=MatchLowercase}
  \defaultfontfeatures[\rmfamily]{Ligatures=TeX,Scale=1}
\fi
% Use upquote if available, for straight quotes in verbatim environments
\IfFileExists{upquote.sty}{\usepackage{upquote}}{}
\IfFileExists{microtype.sty}{% use microtype if available
  \usepackage[]{microtype}
  \UseMicrotypeSet[protrusion]{basicmath} % disable protrusion for tt fonts
}{}
\makeatletter
\@ifundefined{KOMAClassName}{% if non-KOMA class
  \IfFileExists{parskip.sty}{%
    \usepackage{parskip}
  }{% else
    \setlength{\parindent}{0pt}
    \setlength{\parskip}{6pt plus 2pt minus 1pt}}
}{% if KOMA class
  \KOMAoptions{parskip=half}}
\makeatother
\usepackage{xcolor}
\IfFileExists{xurl.sty}{\usepackage{xurl}}{} % add URL line breaks if available
\IfFileExists{bookmark.sty}{\usepackage{bookmark}}{\usepackage{hyperref}}
\hypersetup{
  pdftitle={15-Otras distribuciones},
  pdfauthor={Adrian},
  hidelinks,
  pdfcreator={LaTeX via pandoc}}
\urlstyle{same} % disable monospaced font for URLs
\usepackage[margin=1in]{geometry}
\usepackage{longtable,booktabs,array}
\usepackage{calc} % for calculating minipage widths
% Correct order of tables after \paragraph or \subparagraph
\usepackage{etoolbox}
\makeatletter
\patchcmd\longtable{\par}{\if@noskipsec\mbox{}\fi\par}{}{}
\makeatother
% Allow footnotes in longtable head/foot
\IfFileExists{footnotehyper.sty}{\usepackage{footnotehyper}}{\usepackage{footnote}}
\makesavenoteenv{longtable}
\usepackage{graphicx}
\makeatletter
\def\maxwidth{\ifdim\Gin@nat@width>\linewidth\linewidth\else\Gin@nat@width\fi}
\def\maxheight{\ifdim\Gin@nat@height>\textheight\textheight\else\Gin@nat@height\fi}
\makeatother
% Scale images if necessary, so that they will not overflow the page
% margins by default, and it is still possible to overwrite the defaults
% using explicit options in \includegraphics[width, height, ...]{}
\setkeys{Gin}{width=\maxwidth,height=\maxheight,keepaspectratio}
% Set default figure placement to htbp
\makeatletter
\def\fps@figure{htbp}
\makeatother
\setlength{\emergencystretch}{3em} % prevent overfull lines
\providecommand{\tightlist}{%
  \setlength{\itemsep}{0pt}\setlength{\parskip}{0pt}}
\setcounter{secnumdepth}{-\maxdimen} % remove section numbering
\ifLuaTeX
  \usepackage{selnolig}  % disable illegal ligatures
\fi

\begin{document}
\maketitle

\hypertarget{otras-distribuciones-importantes}{%
\subsubsection{Otras distribuciones
importantes}\label{otras-distribuciones-importantes}}

\begin{itemize}
\tightlist
\item
  La distribución \(\chi^2_k\), donde \(k\) representa los grados de
  libertad de la misma y que procede de la suma de los cuadrados de
  \(k\) distribuciones normales estándar independientes:
\end{itemize}

\[X = Z_1^2 + Z_2^2+\cdots + Z_k^2\sim \chi_k^2\]

\hypertarget{otras-distribuciones-importantes-1}{%
\subsubsection{Otras distribuciones
importantes}\label{otras-distribuciones-importantes-1}}

\begin{itemize}
\tightlist
\item
  La distribución \(t_k\) surge del problema de estimar la media de una
  población normalmente distribuida cuando el tamaño de la muestra es
  pequeña y procede del cociente
\end{itemize}

\[T = \frac{Z}{\sqrt{\chi^2_k/k}}\sim T_k\]

\hypertarget{otras-distribuciones-importantes-2}{%
\subsubsection{Otras distribuciones
importantes}\label{otras-distribuciones-importantes-2}}

\begin{itemize}
\tightlist
\item
  La distribución \(F_{n_1,n_2}\) aparece frecuentemente como la
  distribución nula de una prueba estadística, especialmente en el
  análisis de varianza. Viene definida como el cociente
\end{itemize}

\[F = \frac{\chi^2_{n_1}/n_1}{\chi^2_{n_2}/n_2}\sim F_{n_1,n_2}\]

\hypertarget{distribuciones-continuas-en-r}{%
\subsubsection{Distribuciones continuas en
R}\label{distribuciones-continuas-en-r}}

\begin{longtable}[]{@{}llll@{}}
\toprule
Distribución & Instrucción en R & Instrucción en Python & Parámetros \\
\midrule
\endhead
Uniforme & \texttt{unif} & \texttt{scipy.stats.uniform} & mínimo y
máximo \\
Exponencial & \texttt{exp} & \texttt{scipy.stats.expon} & \(\lambda\) \\
Normal & \texttt{norm} & \texttt{scipy.stats.normal} & media \(\mu\),
desviación típica \(\sigma\) \\
Khi cuadrado & \texttt{chisq} & \texttt{scipy.stats.chi2} & grados de
libertad \\
t de Student & \texttt{t} & \texttt{scipy.stats.t} & grados de
libertad \\
F de Fisher & \texttt{f} & \texttt{scipy.stats.f} & los dos grados de
libertad \\
\bottomrule
\end{longtable}

\hypertarget{otras-distribuciones-conocidas}{%
\subsection{Otras distribuciones
conocidas}\label{otras-distribuciones-conocidas}}

\begin{itemize}
\tightlist
\item
  Distribución de Pareto (Power Law)
\item
  Distribución Gamma y Beta
\item
  Distribución Log Normal
\item
  Distribución de Weibull
\item
  Distribución de Cauchy
\item
  Distribución Exponencial Normal
\item
  Distribución Von Mises
\item
  Distribución Rayleigh
\item
  \ldots{}
\end{itemize}

\end{document}
