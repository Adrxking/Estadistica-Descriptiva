% Options for packages loaded elsewhere
\PassOptionsToPackage{unicode}{hyperref}
\PassOptionsToPackage{hyphens}{url}
%
\documentclass[
]{article}
\title{08-Distribucion Hipergeométrica}
\author{Adrian}
\date{24/1/2022}

\usepackage{amsmath,amssymb}
\usepackage{lmodern}
\usepackage{iftex}
\ifPDFTeX
  \usepackage[T1]{fontenc}
  \usepackage[utf8]{inputenc}
  \usepackage{textcomp} % provide euro and other symbols
\else % if luatex or xetex
  \usepackage{unicode-math}
  \defaultfontfeatures{Scale=MatchLowercase}
  \defaultfontfeatures[\rmfamily]{Ligatures=TeX,Scale=1}
\fi
% Use upquote if available, for straight quotes in verbatim environments
\IfFileExists{upquote.sty}{\usepackage{upquote}}{}
\IfFileExists{microtype.sty}{% use microtype if available
  \usepackage[]{microtype}
  \UseMicrotypeSet[protrusion]{basicmath} % disable protrusion for tt fonts
}{}
\makeatletter
\@ifundefined{KOMAClassName}{% if non-KOMA class
  \IfFileExists{parskip.sty}{%
    \usepackage{parskip}
  }{% else
    \setlength{\parindent}{0pt}
    \setlength{\parskip}{6pt plus 2pt minus 1pt}}
}{% if KOMA class
  \KOMAoptions{parskip=half}}
\makeatother
\usepackage{xcolor}
\IfFileExists{xurl.sty}{\usepackage{xurl}}{} % add URL line breaks if available
\IfFileExists{bookmark.sty}{\usepackage{bookmark}}{\usepackage{hyperref}}
\hypersetup{
  pdftitle={08-Distribucion Hipergeométrica},
  pdfauthor={Adrian},
  hidelinks,
  pdfcreator={LaTeX via pandoc}}
\urlstyle{same} % disable monospaced font for URLs
\usepackage[margin=1in]{geometry}
\usepackage{color}
\usepackage{fancyvrb}
\newcommand{\VerbBar}{|}
\newcommand{\VERB}{\Verb[commandchars=\\\{\}]}
\DefineVerbatimEnvironment{Highlighting}{Verbatim}{commandchars=\\\{\}}
% Add ',fontsize=\small' for more characters per line
\usepackage{framed}
\definecolor{shadecolor}{RGB}{248,248,248}
\newenvironment{Shaded}{\begin{snugshade}}{\end{snugshade}}
\newcommand{\AlertTok}[1]{\textcolor[rgb]{0.94,0.16,0.16}{#1}}
\newcommand{\AnnotationTok}[1]{\textcolor[rgb]{0.56,0.35,0.01}{\textbf{\textit{#1}}}}
\newcommand{\AttributeTok}[1]{\textcolor[rgb]{0.77,0.63,0.00}{#1}}
\newcommand{\BaseNTok}[1]{\textcolor[rgb]{0.00,0.00,0.81}{#1}}
\newcommand{\BuiltInTok}[1]{#1}
\newcommand{\CharTok}[1]{\textcolor[rgb]{0.31,0.60,0.02}{#1}}
\newcommand{\CommentTok}[1]{\textcolor[rgb]{0.56,0.35,0.01}{\textit{#1}}}
\newcommand{\CommentVarTok}[1]{\textcolor[rgb]{0.56,0.35,0.01}{\textbf{\textit{#1}}}}
\newcommand{\ConstantTok}[1]{\textcolor[rgb]{0.00,0.00,0.00}{#1}}
\newcommand{\ControlFlowTok}[1]{\textcolor[rgb]{0.13,0.29,0.53}{\textbf{#1}}}
\newcommand{\DataTypeTok}[1]{\textcolor[rgb]{0.13,0.29,0.53}{#1}}
\newcommand{\DecValTok}[1]{\textcolor[rgb]{0.00,0.00,0.81}{#1}}
\newcommand{\DocumentationTok}[1]{\textcolor[rgb]{0.56,0.35,0.01}{\textbf{\textit{#1}}}}
\newcommand{\ErrorTok}[1]{\textcolor[rgb]{0.64,0.00,0.00}{\textbf{#1}}}
\newcommand{\ExtensionTok}[1]{#1}
\newcommand{\FloatTok}[1]{\textcolor[rgb]{0.00,0.00,0.81}{#1}}
\newcommand{\FunctionTok}[1]{\textcolor[rgb]{0.00,0.00,0.00}{#1}}
\newcommand{\ImportTok}[1]{#1}
\newcommand{\InformationTok}[1]{\textcolor[rgb]{0.56,0.35,0.01}{\textbf{\textit{#1}}}}
\newcommand{\KeywordTok}[1]{\textcolor[rgb]{0.13,0.29,0.53}{\textbf{#1}}}
\newcommand{\NormalTok}[1]{#1}
\newcommand{\OperatorTok}[1]{\textcolor[rgb]{0.81,0.36,0.00}{\textbf{#1}}}
\newcommand{\OtherTok}[1]{\textcolor[rgb]{0.56,0.35,0.01}{#1}}
\newcommand{\PreprocessorTok}[1]{\textcolor[rgb]{0.56,0.35,0.01}{\textit{#1}}}
\newcommand{\RegionMarkerTok}[1]{#1}
\newcommand{\SpecialCharTok}[1]{\textcolor[rgb]{0.00,0.00,0.00}{#1}}
\newcommand{\SpecialStringTok}[1]{\textcolor[rgb]{0.31,0.60,0.02}{#1}}
\newcommand{\StringTok}[1]{\textcolor[rgb]{0.31,0.60,0.02}{#1}}
\newcommand{\VariableTok}[1]{\textcolor[rgb]{0.00,0.00,0.00}{#1}}
\newcommand{\VerbatimStringTok}[1]{\textcolor[rgb]{0.31,0.60,0.02}{#1}}
\newcommand{\WarningTok}[1]{\textcolor[rgb]{0.56,0.35,0.01}{\textbf{\textit{#1}}}}
\usepackage{graphicx}
\makeatletter
\def\maxwidth{\ifdim\Gin@nat@width>\linewidth\linewidth\else\Gin@nat@width\fi}
\def\maxheight{\ifdim\Gin@nat@height>\textheight\textheight\else\Gin@nat@height\fi}
\makeatother
% Scale images if necessary, so that they will not overflow the page
% margins by default, and it is still possible to overwrite the defaults
% using explicit options in \includegraphics[width, height, ...]{}
\setkeys{Gin}{width=\maxwidth,height=\maxheight,keepaspectratio}
% Set default figure placement to htbp
\makeatletter
\def\fps@figure{htbp}
\makeatother
\setlength{\emergencystretch}{3em} % prevent overfull lines
\providecommand{\tightlist}{%
  \setlength{\itemsep}{0pt}\setlength{\parskip}{0pt}}
\setcounter{secnumdepth}{-\maxdimen} % remove section numbering
\ifLuaTeX
  \usepackage{selnolig}  % disable illegal ligatures
\fi

\begin{document}
\maketitle

\hypertarget{distribucion-hipergeometrica}{%
\subsection{Distribucion
Hipergeometrica}\label{distribucion-hipergeometrica}}

Consideremos el experimentro ``Extrar a la vez n objetos donde hay N de
tipo A y M de tipo B''. Si X es v.a. que mide el ``numero de objetos del
tipo A'', diremos que X se distribuye como una Hipergeometrica con
parametros N,M,n. \[X\sim \text{H}(N,M,n)\]

\begin{itemize}
\item
  El \textbf{dominio} de \(X\) será \(D_X = \{0,1,2,\dots,N\}\) (en
  general)
\item
  La \textbf{función de probabilidad} vendrá dada por
  \[f(k) = \frac{{N\choose k}{M\choose n-k}}{N+M\choose n}\]
\item
  La \textbf{función de distribución} vendrá dada por \[F(x) = \left\{
  \begin{array}{cl}
     0 & \text{si } x<0 
  \\ \sum_{k=0}^xf(k) & \text{si } 0\le x<n
  \\ 1 & \text{si } x\ge n
  \end{array}
  \right.\]
\item
  \textbf{Esperanza} \(E(X) = \frac{nN}{N+M}\)
\item
  \textbf{Varianza}
  \(Var(X) = \frac{nNM}{(N+M)^2}\cdot\frac{N+M-n}{N+M-1}\)
\end{itemize}

\hypertarget{paqueteria}{%
\subsubsection{Paqueteria}\label{paqueteria}}

\begin{itemize}
\item
  En \texttt{R} tenemos las funciones del paquete \texttt{Rlab}:
\item
  dhyper(x, m, n, k)
\item
  phyper(q, m, n, k)
\item
  qhyper(p, m, n, k)
\item
  rhyper(nn, m, n, k) donde \texttt{m} es el número de objetos del
  primer tipo, \texttt{n} el número de objetos del segundo tipo y
  \texttt{k} el número de extracciones realizadas.
\item
  En \texttt{Python} tenemos las funciones del paquete
  \texttt{scipy.stats.hypergeom}:
\item
  pmf(k,M, n, N)
\item
  cdf(k,M, n, N)
\item
  ppf(q,M, n, N)
\item
  rvs(M, n, N, size) donde \texttt{M} es el número de objetos del primer
  tipo, \texttt{N} el número de objetos del segundo tipo y \texttt{n} el
  número de extracciones realizadas.
\end{itemize}

\hypertarget{ejemplo}{%
\subsubsection{Ejemplo}\label{ejemplo}}

Supongamos que tenemos 20 animales, de los cuales 7 son perros. Queremos
medir la probabilidad de encontrar un numero determinado de perros si
elegimos \(x=12\) animales al azar.

\begin{Shaded}
\begin{Highlighting}[]
\FunctionTok{library}\NormalTok{(Rlab)}
\end{Highlighting}
\end{Shaded}

\begin{verbatim}
## Rlab 2.15.1 attached.
\end{verbatim}

\begin{verbatim}
## 
## Attaching package: 'Rlab'
\end{verbatim}

\begin{verbatim}
## The following objects are masked from 'package:stats':
## 
##     dexp, dgamma, dweibull, pexp, pgamma, pweibull, qexp, qgamma,
##     qweibull, rexp, rgamma, rweibull
\end{verbatim}

\begin{verbatim}
## The following object is masked from 'package:datasets':
## 
##     precip
\end{verbatim}

\begin{Shaded}
\begin{Highlighting}[]
\NormalTok{M }\OtherTok{=} \DecValTok{7}
\NormalTok{N }\OtherTok{=} \DecValTok{13}
\NormalTok{k }\OtherTok{=} \DecValTok{12}
\FunctionTok{dhyper}\NormalTok{(}\AttributeTok{x =} \DecValTok{0}\SpecialCharTok{:}\DecValTok{12}\NormalTok{, }\AttributeTok{m =}\NormalTok{ M, }\AttributeTok{n =}\NormalTok{ N, }\AttributeTok{k =}\NormalTok{ k)}
\end{Highlighting}
\end{Shaded}

\begin{verbatim}
##  [1] 0.0001031992 0.0043343653 0.0476780186 0.1986584107 0.3575851393
##  [6] 0.2860681115 0.0953560372 0.0102167183 0.0000000000 0.0000000000
## [11] 0.0000000000 0.0000000000 0.0000000000
\end{verbatim}

\begin{Shaded}
\begin{Highlighting}[]
\FunctionTok{phyper}\NormalTok{(}\AttributeTok{q =} \DecValTok{0}\SpecialCharTok{:}\DecValTok{12}\NormalTok{, }\AttributeTok{m =}\NormalTok{ M, }\AttributeTok{n =}\NormalTok{ N, }\AttributeTok{k =}\NormalTok{ k)}
\end{Highlighting}
\end{Shaded}

\begin{verbatim}
##  [1] 0.0001031992 0.0044375645 0.0521155831 0.2507739938 0.6083591331
##  [6] 0.8944272446 0.9897832817 1.0000000000 1.0000000000 1.0000000000
## [11] 1.0000000000 1.0000000000 1.0000000000
\end{verbatim}

\begin{Shaded}
\begin{Highlighting}[]
\FunctionTok{qhyper}\NormalTok{(}\AttributeTok{p =} \FloatTok{0.5}\NormalTok{, }\AttributeTok{m =}\NormalTok{ M, }\AttributeTok{n =}\NormalTok{ N, }\AttributeTok{k =}\NormalTok{ k)}
\end{Highlighting}
\end{Shaded}

\begin{verbatim}
## [1] 4
\end{verbatim}

\begin{Shaded}
\begin{Highlighting}[]
\FunctionTok{rhyper}\NormalTok{(}\AttributeTok{nn =} \DecValTok{1000}\NormalTok{, }\AttributeTok{m =}\NormalTok{ M, }\AttributeTok{n =}\NormalTok{ N, }\AttributeTok{k =}\NormalTok{ k) }\OtherTok{{-}\textgreater{}}\NormalTok{ data}
\FunctionTok{hist}\NormalTok{(data, }\AttributeTok{breaks =} \DecValTok{8}\NormalTok{)}
\end{Highlighting}
\end{Shaded}

\includegraphics{08-Distribucion-Hipergeométrica_files/figure-latex/unnamed-chunk-1-1.pdf}

\end{document}
